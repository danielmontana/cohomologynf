\documentclass[a4paper,12pt, leqno]{article}
\usepackage{amsmath,amsthm,amssymb,enumitem, cclicenses, hyperref,amsfonts,tikz-cd,mathrsfs, tikz}
\usetikzlibrary{matrix, calc, arrows}
\setlength{\parindent}{0cm}
\setlength{\parskip}{6pt}
\tikzset{
	curvedlink/.style={
		to path={
			let \p1=(\tikztostart.east), \p2=(\tikztotarget.west),
			\n1= {abs(\y2-\y1)/4} in
			(\p1) arc(90:-90:\n1) -- ([yshift=2*\n1]\p2) arc (90:270:\n1)
		},
	}
}
\newcommand{\homo}{\mathrm{Hom}}
\newcommand{\Q}{\mathbb{Q}}
\newcommand{\Z}{\mathbb{Z}}
\newcommand{\im}{\mathrm{im}\,}
\newcommand{\id}{\mathrm{id}}
\DeclareMathOperator{\coker}{coker}

\newtheoremstyle{defprop} % name
{\topsep}                    % Space above
{\topsep}                    % Space below
{\mdseries}                  % Body font
{}                           % Indent amount
{\bfseries}                   % Theorem head font
{.}                          % Punctuation after theorem head
{.5em}                       % Space after theorem head
{}  % Theorem head spec (can be left empty, meaning ‘normal’)
 
\newtheoremstyle{dem} % name
{\topsep}                    % Space above
{\topsep}                    % Space below
{\mdseries}                  % Body font
{}                           % Indent amount
{\itshape}                   % Theorem head font
{.}                          % Punctuation after theorem head
{.5em}                       % Space after theorem head
{}  % Theorem head spec (can be left empty, meaning ‘normal’)          

\newtheoremstyle{prop} % name
{\topsep}                    % Space above
{\topsep}                    % Space below
{\itshape}                  % Body font
{}                           % Indent amount
{\bfseries}                   % Theorem head font
{.}                          % Punctuation after theorem head
{.5em}                       % Space after theorem head
{}  % Theorem head spec (can be left empty, meaning ‘normal’)          

\theoremstyle{defprop}
\newtheorem{defprop}{Definición/Proposición}
\newtheorem{definicion}{Definición}
\newtheorem*{ejemplo}{Ejemplo}

\theoremstyle{prop}
\newtheorem{prop}{Proposición}

\theoremstyle{dem}
\newtheorem*{dem}{Demostración}

%Flechas
\newcommand\rightthreearrow{%
	\mathrel{\vcenter{\mathsurround0pt
			\ialign{##\crcr
				\noalign{\nointerlineskip}$\rightarrow$\crcr
				\noalign{\nointerlineskip}$\rightarrow$\crcr
				\noalign{\nointerlineskip}$\rightarrow$\crcr
			}%
	}}%
}

\newcommand\rightfourarrow{%
	\mathrel{\vcenter{\mathsurround0pt
			\ialign{##\crcr
				\noalign{\nointerlineskip}$\rightarrow$\crcr
				\noalign{\nointerlineskip}$\rightarrow$\crcr
				\noalign{\nointerlineskip}$\rightarrow$\crcr
				\noalign{\nointerlineskip}$\rightarrow$\crcr
			}%
	}}%
}



\title{Día 3}
\author{Daniel M.}

\begin{document}
	\section{Módulos inducidos (continuación)}
	\begin{prop}
		Los $G$-módulos inducidos $M=\ind (A)$ son cohomológicamente triviales. Si además $G$ es finito, tenemos que $\hat{H}^n(G,M)=0$ para todo $n \in \Z$.
	\end{prop}
	\begin{dem}
		Consideramos las resoluciones estándar $X^{\bullet}(G,A)$ y $X^{\bullet}(G,\ind (A))$, es decir aquellas en las que:
		\begin{equation*}
		X^n=\mathrm{Map}(G^{n+1},A);X^n=\mathrm{Map}(G^{n+1},\ind (A))
		\end{equation*}
		respectivamente. Tenemos una aplicación:
		\begin{equation*}
		X^n(G,\ind (A))^G \rightarrow X^n(G,A)
		\end{equation*}
		dada por $x(\sigma_0,...,\sigma_n)\mapsto y(\sigma_0,...,\sigma_n)=x(\sigma_0,...,\sigma_n)(1)$. De hecho, esto un isomorfismo (ejercicio: encontrar el inverso) luego tenemos un isomorfismo de complejos:
		\begin{equation*}
		C^{\bullet}(G,\ind (A))\cong X^{\bullet}(G,A).
		\end{equation*}
		El primer día demostramos que $X^{\bullet}(G,A)$ era exacta, luego:
		\begin{equation*}
		H^n(G,\ind (A))=H^n(C^{\bullet}(G,\ind (A)))=0
		\end{equation*}
		para $n\geq 1$. Si $H$ es  un subgrupo cerrado de $G$, por la proposición anterior podemos escribir $\ind (A)=\mathrm{Ind}_H(B)$ para algún $B$ y entonces: 
		\begin{equation*}
		H^n(H,\ind (A))=0.
		\end{equation*}
		
		Cuando $G$ es finito, se puede repetir el argumento en el complejo extendido $(X^n)_{n \in \Z}$ para obtener $\hat{H}^n(G, \ind (A))=0$ para todo $n \in \Z$. \qed
	\end{dem}
	Este resultado nos permite aplicar una técnica conocida como \textit{dimension shifting} que consiste en reducir demostraciones sobre todos los grupos de cohomología a una única dimension. Dado $A$, definimos $A_1$ con la siguiente sucesión exacta:
	\begin{center}
		\begin{tikzcd}
		0 \arrow{r}& A \arrow{r}{i} & \ind (A) \arrow{r} & A_1 \arrow{r} & 0,
		\end{tikzcd}
	\end{center}
	donde $ia$ es la función constante $ia(\sigma)=a$. Si $H\leq G$ es un subgrupo cerrado, por la proposición anterior tenemos que el homomorfismo:
	\begin{equation*}
	\delta: H^n(H,A_1) \rightarrow H^{n+1}(H,A)
	\end{equation*} 
	es sobreyectivo para $n=0$ y un isomorfismo para $n>0$. Aplicando el mismo proceso inductivamente para $A_0 = A$ y $A_+=(A_{p-1})_1$ para $p>0$ obtenemos:
	\begin{prop}\label{dimsh}
		Para $n,p \geq0$ y cualquier subgrupo $H$ de $G$, tenemos un homomorfismo canónico:\begin{equation*}
		\delta^p: H^n(H,A_p) \rightarrow H^{n+p}(H,A)
		\end{equation*}
		que es sobreyectivo para $n=0$ y un isomorfismo para $n>0$.
	\end{prop}
	
	Si $G$ es un grupo finito, también podemos obtener un resultado parecido para la cohomología de Tate. Consideremos la sucesión exacta:
	\begin{center}
		\begin{tikzcd}
		0 \arrow{r}& A_{-1} \arrow{r}& \ind (A) \arrow{r}{\nu} & A \arrow{r} &0,
		\end{tikzcd}
	\end{center}
	donde $\nu:x \mapsto \sum_{g \in G} x(g)$. Definimos $A_p=(A_{p+1})_{-1}$ para $p<0$; utilizando que $\ind (A) \cong A \otimes \Z[G]$ es fácil ver que:
	\begin{center}
		$A_p \cong A \otimes J_G^{\otimes p}$ y $A_{-p}\cong I_G^{\otimes p}$
	\end{center}
	para $p\geq 0$, donde $I_G,J_G$ vienen dados por las sucesiones:
	\begin{center}
		\begin{tikzcd}
		0 \arrow{r} & I_G \arrow{r} & \Z[G] \arrow{r}{\epsilon} & \Z \arrow{r}& 0,\\
		0 \arrow{r}& \Z \arrow{r}{N_G}&\Z[G] \arrow{r} & J_G \arrow{r} & 0.
		\end{tikzcd}
	\end{center}
	A $\epsilon$ se le denomina \textit{augmentation map}:
	\begin{equation*}
	\epsilon: \sum_{\sigma \in G} a_{\sigma} \sigma \mapsto \sum_{\sigma \in G} a_{\sigma},
	\end{equation*}
	y al $G$-módulo $I_G$ se le llama \textit{augmentation ideal}. Como $\hat{H}^n(H,\ind (A))=0$, obtenemos isomorfismos canónicos:
	\begin{equation*}
	\hat{H}^n(H,A)\cong \hat{H}^{n-p}(H,A_p)
	\end{equation*}
	para todo $n,p \in \Z$.
	
	Volviendo a caso general de un grupo profinito $G$, otra forma de calcular los grupos de cohomología es utilizando resoluciones acíclicas $Y^{\bullet}$ de $A$, en cuyo caso $H^n(G,A)\cong H^n(H^0(G,Y^{\bullet}))$ (ver Proposición 1.3.9).
	\section{El producto cup}
	Recordamos que dados dos $G$-módulos $A$ y $B$, $A \otimes_{\mathbb{Z}} B$ con la acción diagonal también es un $G$-módulo. Esto nos permite definir para $p,q \geq 0 $:
	\begin{center}
		\begin{tikzcd}
		C^p(G,A)\times C^q(G,B) \arrow{r}{\cup}&  C^{p+q}(G, A \otimes B)
		\end{tikzcd}
	\end{center}
	dado por:
	\begin{equation*}
	(a\cup b)(\sigma_0,...,\sigma_{p+q})=a(\sigma_0,...,\sigma_q)\otimes b(\sigma_p,...,\sigma_{p+q}).
	\end{equation*}
	Esta función verifica la siguiente formula:
	\begin{equation*}
	\partial(a \cup b)=(\partial a) \cup b + (-1)^p (a\cup \partial b),
	\end{equation*}
	que puede verificarse con un simple cálculo. Claramente, si $a$ y $b$ son cociclos entonces $a \cup b$ también es un cociclo. Además, si uno es  un cociclo y el otro es un coborde, $a \cup b$ también es un coborde. En definitiva, $\cup$ induce una aplicación bilineal:
	\begin{center}
		\begin{tikzcd}
		H^p(G,A)\times H^q(G,B) \arrow{r}{\cup}&  H^{p+q}(G, A \otimes B),
		\end{tikzcd}
	\end{center}
	dado por $(\alpha, \beta) \mapsto \alpha \cup \beta$. A esta aplicación le llamamos \textbf{producto cup}. También se le llama producto cup a la composición:
	\begin{center}
		\begin{tikzcd}
		H^p(G,A) \times H^q(G,B) \arrow{r}{\cup} & H^{p+q}(G, A \otimes B)  \arrow{r} & H^{p+q}(G,C),
		\end{tikzcd}
	\end{center}
	que viene inducida por una aplicación bilineal $A \times B \rightarrow C$ que factoriza en el producto tensorial. 
	
	Cada vez que definimos una aplicación nueva en la cohomología, tenemos que verificar sus propiedades functoriales y su compatibilidad con las anteriores. Directamente de la definición se sigue que el producto cup conmuta con homomorfismos $A \rightarrow A', B \rightarrow B'$. A continuación demostramos la compatibilidad con $\delta$:
	\begin{prop}
		Sean
		
		\begin{center}
			$0 \rightarrow A' \rightarrow A \rightarrow A'' \rightarrow 0$ y $0 \rightarrow C' \rightarrow C \rightarrow C''\rightarrow 0$.
		\end{center}
		dos sucesiones exactas de $G$-módulos. Sea $B$ otro $G$-módulo y supongamos que existe un emparejamiento $A \times B \rightarrow C$ que induce $A' \times B\rightarrow C'$ y $A'' \times B \rightarrow C''$.  Entonces el diagrama siguiente conmuta:
		
		\begin{center}
			\begin{tikzcd}[column sep=small]
			H^p(G,A'') \arrow[d,"\delta"] \arrow[r,phantom,"\times"] & H^q(G,B) \arrow[d,"id"] \arrow[r,"\cup"]& H^{p+q}(G,C'') \arrow[d,"\delta"]\\
			H^{p+1}(G,A') \arrow[r,phantom,"\times"]& H^q(G,B)  \arrow[r,"\cup"]&  H^{p+q+1}(G,C')
			\end{tikzcd}
		\end{center}
		Es decir, $\delta(\alpha'' \cup \beta)=\delta \alpha'' \cup \beta$.
		
		Análogamente, dadas dos sucesiones exactas:
		\begin{center}
			$0 \rightarrow B' \rightarrow B \rightarrow B'' \rightarrow 0$ y $0 \rightarrow C' \rightarrow C \rightarrow C''\rightarrow 0$
		\end{center}
		con un emparejamiento $A \times B \rightarrow C$ que induce $A \times B' \rightarrow C'$ y $A \times B'' \rightarrow C''$, el diagrama siguiente conmuta:
		\begin{center}
			\begin{tikzcd}[column sep=small]
			H^p(G,A) \arrow[d,"id"] \arrow[r,phantom,"\times"] & H^q(G,B'') \arrow[d,"\delta"] \arrow[r,"\cup"]& H^{p+q}(G,C'') \arrow[d,"(-1)^p\delta"]\\
			H^{p}(G,A) \arrow[r,phantom,"\times"]& H^{q+1}(G,B')  \arrow[r,"\cup"]&  H^{p+q+1}(G,C').
			\end{tikzcd}
		\end{center}
		Es decir, $(-1)^p \delta(\alpha \cup \beta'') = \alpha \cup \delta \beta''$.
	\end{prop}
	\begin{dem}
		Demostramos la primera igualdad, siendo análoga la demostración de la segunda. Sean $\alpha''=\overline{a''},\beta=\overline{b}$ para $a''\in Z^p(G,A''),b\in Z^q(G,A)$. 
		
		El functor $A \mapsto C^p(G,A)$ es exacto, luego podemos elegir $a\in C^p(G,A)$ que esté en la preimagen de $a''$.	Entonces por definición de $\delta$, $\delta \alpha''\in H^{p+1}(G,A')$ es representado por $\partial a\in Z^{p+1}(G,A')$  (identificando $A'$ con su imagen en $A$).
		
		A su vez, $\delta(\alpha'' \cup \beta)$ es representado por $\partial(a \cup b)=\partial a \cup b$, ya que $\partial b = 0$. Pasando a cohomología esto significa:
		\begin{equation*}
		\delta(\alpha'' \cup \beta)=\delta \alpha'' \cup \beta.
		\end{equation*}\qed
	\end{dem}
	\begin{prop}
		El producto cup verifica:
		
		\begin{enumerate}[label = \roman*)]
			\item $(\alpha \cup \beta) \cup \gamma = \alpha \cup (\beta \cup \gamma)$.
			\item $\alpha \cup \beta = (-1)^{pq}(\beta \cup \alpha)$.
		\end{enumerate}
	\end{prop}
	\begin{dem}
		La primera afirmación es una comprobación directa. Para la segunda utilizaremos el método de \textit{dimension shifting} introducido en la Proposición \ref{dimsh}. Recordamos que existen homomorfismos sobreyectivos $\delta^n: H^0(G,A_n)\rightarrow H^n(G,A).$ Aplicando la proposición anterior $p$ y $q$ veces respectivamente obtenemos un diagrama conmutativo:
		\begin{center}
			\begin{tikzcd}
			[column sep=small]
			H^0(G,A_p) \arrow[d,"\delta^p"] \arrow[r,phantom,"\times"] & H^0(G,B_q) \arrow[d,"id"] \arrow[r,"\cup"]& H^0(G,(A\otimes B_q)_p) \arrow[d,"\delta^p"] \arrow[r,phantom,"="]& H^0(G,A_p \otimes B_q)\\
			H^p(G,A') \arrow[r,phantom,"\times"] \arrow[d,"id"]& H^0(G,B_q)  \arrow[r,"\cup"] \arrow[d,"\delta^p"]&  H^p(G,(A \otimes B)_q) \arrow[d,"(-1)^{pq}\delta^q"] \arrow[r,phantom,"="] & H^p(G,A \otimes B_q)\\
			H^p(G,A\arrow[r,phantom,"\times"] & H^q(G,B) \arrow[r,"\cup"] & H^{p+q}(G,A\otimes B)
			\end{tikzcd}	
		\end{center}
		Para $p=q=0$ la identidad es trivial. Como las flechas verticales son sobreyectivas, obtenemos $\alpha \cup B=(-1)^{pq}(\beta \cup \alpha)$ para $p,q \geq 0$.\qed
	\end{dem}
	%\begin{prop}
	%	Sean:
	%	\begin{center}
	%		$0 \rightarrow A' \rightarrow A \rightarrow A'' \rightarrow 0$ y $0 \rightarrow B' \rightarrow B \rightarrow B'' \rightarrow 0$
	%	\end{center}
	%dos sucesiones exactas de $G$-módulos tal que existe un emparejamiento $\phi: A \times B \rightarrow C$ con $C$ otro $G$-módulo y tal que $\phi(A' %\times B')=0$. Es decir, $\phi$ induce $\phi',\phi''$ tal que el siguiente diagrama conmuta:
	%\begin{center}
	%	\begin{tikzcd}
	%	A' \arrow[r,phantom,"\times"] \arrow[d,hook,"i"] & B'' \arrow[r,"\phi'"] & C \arrow[d,equal]\\
	%	A \arrow[d,two heads,"j"] \arrow[r,phantom,"\times"] & B \arrow[u, two heads, "v"] \arrow[r,"\phi"]& C \arrow[d,equal]\\
	%	A'' \arrow[r,phantom,"\times"] & B' \arrow[u,hook,"u"] \arrow[r,"\phi''"]& C
	%	\end{tikzcd}
	%\end{center}
	%\end{prop}
	
	
	El producto cup también se puede definir en dimensiones arbitrarias cuando $G$ es finito (es decir, para cohomología de Tate) y de manera que los resultados que hemos demostrado también se cumplan. Esto se puede consultar en la Proposición 1.4.7 del libro. 
	
\end{document}