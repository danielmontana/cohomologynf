\documentclass[a4paper,12pt, leqno]{article}
\usepackage{amsmath,amsthm,amssymb,enumitem, cclicenses, hyperref,amsfonts,tikz-cd,mathrsfs, tikz}
\usetikzlibrary{matrix, calc, arrows}
\setlength{\parindent}{0cm}
\setlength{\parskip}{6pt}
\tikzset{
	curvedlink/.style={
		to path={
			let \p1=(\tikztostart.east), \p2=(\tikztotarget.west),
			\n1= {abs(\y2-\y1)/4} in
			(\p1) arc(90:-90:\n1) -- ([yshift=2*\n1]\p2) arc (90:270:\n1)
		},
	}
}
\newcommand{\homo}{\mathrm{Hom}}
\newcommand{\Q}{\mathbb{Q}}
\newcommand{\Z}{\mathbb{Z}}
\newcommand{\im}{\mathrm{im}\,}
\newcommand{\id}{\mathrm{id}}
\DeclareMathOperator{\coker}{coker}

\newtheoremstyle{defprop} % name
{\topsep}                    % Space above
{\topsep}                    % Space below
{\mdseries}                  % Body font
{}                           % Indent amount
{\bfseries}                   % Theorem head font
{.}                          % Punctuation after theorem head
{.5em}                       % Space after theorem head
{}  % Theorem head spec (can be left empty, meaning ‘normal’)
 
\newtheoremstyle{dem} % name
{\topsep}                    % Space above
{\topsep}                    % Space below
{\mdseries}                  % Body font
{}                           % Indent amount
{\itshape}                   % Theorem head font
{.}                          % Punctuation after theorem head
{.5em}                       % Space after theorem head
{}  % Theorem head spec (can be left empty, meaning ‘normal’)          

\newtheoremstyle{prop} % name
{\topsep}                    % Space above
{\topsep}                    % Space below
{\itshape}                  % Body font
{}                           % Indent amount
{\bfseries}                   % Theorem head font
{.}                          % Punctuation after theorem head
{.5em}                       % Space after theorem head
{}  % Theorem head spec (can be left empty, meaning ‘normal’)          

\theoremstyle{defprop}
\newtheorem{defprop}{Definición/Proposición}
\newtheorem{definicion}{Definición}
\newtheorem*{ejemplo}{Ejemplo}

\theoremstyle{prop}
\newtheorem{prop}{Proposición}

\theoremstyle{dem}
\newtheorem*{dem}{Demostración}

%Flechas
\newcommand\rightthreearrow{%
	\mathrel{\vcenter{\mathsurround0pt
			\ialign{##\crcr
				\noalign{\nointerlineskip}$\rightarrow$\crcr
				\noalign{\nointerlineskip}$\rightarrow$\crcr
				\noalign{\nointerlineskip}$\rightarrow$\crcr
			}%
	}}%
}

\newcommand\rightfourarrow{%
	\mathrel{\vcenter{\mathsurround0pt
			\ialign{##\crcr
				\noalign{\nointerlineskip}$\rightarrow$\crcr
				\noalign{\nointerlineskip}$\rightarrow$\crcr
				\noalign{\nointerlineskip}$\rightarrow$\crcr
				\noalign{\nointerlineskip}$\rightarrow$\crcr
			}%
	}}%
}



\begin{document}
\section{Extensiones}
En esta sección vamos a ver la relación de los grupos de cohomología de dimensión 1 con el problema de clasificar ciertas sucesiones exactas cortas. 

\begin{definicion}
	Sean $(G,\cdot)$ un grupo finito y $(A,+)$ un grupo abeliano. Decimos que $E$ es una extensión de $G$ por $A$ si existe una sucesión exacta corta de $\mathbb{Z}$-módulos:
\begin{center}
		\begin{tikzcd}
	0 \arrow{r} & A \arrow{r}{i}& E \arrow{r}{\pi}& G \arrow{r}& 0.
	\end{tikzcd}
\end{center}
	\end{definicion}
Nótese que podemos identificar $A$ con el subgrupo normal $i(A) \unlhd E$ tal que $G \simeq E/A$. Es decir, $E$ actúa de forma natural en $A$ por conjugación. Esto nos permite definir a su vez una acción de $G$ en $A$: dado $g \in G$, sea $\overline{g}\in E$ tal que $\pi(\overline{g})=g$. Entonces:
\begin{equation*}
i(g\cdot a):= \overline{g} i(a) \overline{g}^{-1}.
\end{equation*}
Es fácil comprobar que esta acción verifica todos las propiedas que requiere un $G$-módulo. Esto nos permite definir extensiones de $G$ por el $G$-módulo $A$ si acompañamos al grupo $A$ de esta estructura.
\begin{definicion}
Dos extensiones son equivalentes si existe un homomorfismo $\phi:E \rightarrow E'$ tal que el siguiente diagrama conmuta:
\begin{center}
	\begin{tikzcd}
	0 \arrow{r} & A \arrow{r}{i} \ar[d,equal]& E \arrow{r}{\pi} \arrow{d}{\phi}& G \ar[d,equal] \arrow{r}  & 0\\
	0  \arrow{r} & A \arrow{r}{i'}& E' \arrow{r}{\pi'}& G   \arrow{r}& 0 
\end{tikzcd}
\end{center}
En este caso, uno puede comprobar que $\phi$ es un isomorfismo. 
\end{definicion}
\begin{definicion}
Dada una sucesión exacta corta de $R$-módulos: 
\begin{center}
	\begin{tikzcd}0 \arrow{r} & A \arrow{r}{f}& B \arrow{r}{g}& C \arrow{r}& 0, \end{tikzcd}
\end{center}
	decimos que \textit{escinde} si existe un homomorfismo $\sigma:C \rightarrow B$ tal que $g \sigma=\id_C$. A $\sigma$ se le denomina una sección.
	
\end{definicion}
Por ejemplo la siguiente s.e.c. de grupos abelianos no escinde:
\begin{equation*}
0 \rightarrow \Z /2 \rightarrow \Z /4 \rightarrow \Z /2 \rightarrow 0.
\end{equation*}
Sea $E$ una extensión de $G$ por $A$ escindida. Identificando $E$ con $A \times G$ como conjuntos podemos deducir la operación de grupo:
\begin{equation*}
(i(a)\sigma(g))(i(a')\sigma(g'))=i(a)\sigma(g)i(a')\sigma(g)^{-1} \sigma(g) \sigma(g')=i(a)i(ga')\sigma(g) \sigma(g').
\end{equation*}
Es decir, la operación viene dada por: 
\begin{equation*}
(a,g)\cdot(a',g')=(a+ga',gg'),
\end{equation*}
lo cual corresponde al producto semidirecto $A\rtimes G$. Es decir, que si una extensión de $G$ por $A$ escinde, éste es isomorfa a $A \rtimes G$.
\begin{definicion}
	Sean $\sigma, \sigma': G \rightarrow E$ dos secciones. Decimos que son conjugadas si existe $a\in A$ tal que $\sigma(g)=i(a)\sigma'(g)i(a)^{-1}$ para todo $g\in G$.
	\end{definicion}
\begin{prop} Las extensiones de $G$ por $A$ escindidas corresponden a 1-cociclos inhomogéneos. Es decir, a homomorfismos cruzados $x:G \rightarrow A$ tal que $x(gg')=x(g)+gx(g')$. 
	\end{prop}
\begin{dem}
	La sección $G \rightarrow A \rtimes G$ tiene que tener la forma $g \mapsto (x(a),g)$ para algún $x:G \rightarrow A$. Como se tiene que verificar $\sigma(g g')=\sigma(g) \sigma(g')$ tenemos que: 
	\begin{equation*}
	(x(gg'),gg')=(x(g),g)\cdot (x(g'),g')=(x(g)+gx(g'),gg').
	\end{equation*}\qed
	\end{dem}
\begin{prop}
	Dos homomorfismos cruzados correspondientes a secciones conjugadas difieren en un 1-coborde. 
	\end{prop}
\begin{dem}
	Sea $a \in A$ tal que $\sigma(g)i(a)=i(a)\sigma'(g)$, es decir: 
	\begin{equation*}
	(x(g),g)\cdot (a,1)=(a,1)\cdot(x(g'),g').
	\end{equation*}
	Esto implica que $x(g)+ga=a+x'(g)$, luego $x(g)-x'(g)=ga-a$ que es el resultado que queríamos. \qed
	\end{dem}
\paragraph{Conclusión:}
podemos interpretar el grupo $H^1(G,A)$ como las secciones de una extensión escindida de $G$ por $A$ módulo que las secciones sean conjugadas. 
\section{Otros ejemplos}
\begin{teor}
	\end{teor}
\end{document}
